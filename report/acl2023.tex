% This must be in the first 5 lines to tell arXiv to use pdfLaTeX, which is strongly recommended.
\pdfoutput=1
% In particular, the hyperref package requires pdfLaTeX in order to break URLs across lines.

\documentclass[11pt]{article}

% Remove the "review" option to generate the final version.
\usepackage{ACL2023}

% Standard package includes
\usepackage{times}
\usepackage{latexsym}

% For proper rendering and hyphenation of words containing Latin characters (including in bib files)
\usepackage[T1]{fontenc}
% For Vietnamese characters
% \usepackage[T5]{fontenc}
% See https://www.latex-project.org/help/documentation/encguide.pdf for other character sets

% This assumes your files are encoded as UTF8
\usepackage[utf8]{inputenc}

% This is not strictly necessary, and may be commented out.
% However, it will improve the layout of the manuscript,
% and will typically save some space.
\usepackage{microtype}

% This is also not strictly necessary, and may be commented out.
% However, it will improve the aesthetics of text in
% the typewriter font.
\usepackage{inconsolata}

\usepackage{amsmath,amssymb,amsfonts}
%\usepackage{ebgaramond-maths}
\usepackage{algorithmic}
\usepackage{graphicx, subcaption}
\usepackage{textcomp}
\usepackage{xcolor}
\usepackage{float}

\title{Low resource scalar implicature in te reo Māori}

\author{Caleb Moses\\
  \texttt{caleb.moses@mail.mcgill.ca} \\
  \textit{dept. of Computer Science, McGill University}
}

\begin{document}
\pagestyle{plain}
\thispagestyle{plain}

\maketitle
\begin{abstract}
In this paper, I designed an implicature game based on a teaching methodology designed to teach low resource languages in a multi-modal way.
\end{abstract}

%To this end,
% ranging from convicted offenders to regular employees

\section{Introduction}
The Silent Way \cite{gattegno2010teaching} is a language teaching methodology which emphasises the subordination of teaching to learning. Lessons are conducted entirely in the target language, with a minimal vocabulary established in the early stages which incorporates simple utterances, instructions as well as hand-gestures. Cuisenaire rods are used to establish a world-state, which becomes a point for discussion where utterances can be hypothesised by the student and tested. When done in groups, lessons are overseen by a teacher but students take turns demonstrating their understanding to each other, and are able to correct and learn together.

In New Zealand, as well as in other countries, the Silent Way has been adopted by the Māori people as a dominant teaching methodology, especially for second-language learners of the Māori language (te reo Māori). The methodology was specifically adapted to te reo Māori \cite{mataira1980effectiveness,ka2008ngoingoi} by Dame Kāterina Te Heikōkō Mataira and Ngoingoi Pewhairangi where it is referred to as ``Te Ataarangi''. In the classroom, students in Te Ataarangi are encouraged to master grammar by experiencing the language primarily through speaking and listening, and doing practical exercises with other students. The classroom operates according to the following rules:

\begin{itemize}
  \item Speaking in English is not allowed
  \item Belittling other students is not allowed
  \item Students are not allowed to prompt other students
  \item Students can only reply when a question is posed directly to them
  \item Students must practice empathy with their peers
\end{itemize}

Because the classroom proceeds to the next topic only once every student has demonstrated their understanding, students cannot stall for time, expecting the teacher to lose patience and move the class on. Because of this, Te Ataarangi has been effective for many students who do not function well in a traditional classroom.

The experiments in this paper have been designed with the aim of replicating the Te Ataarangi method of teaching using computational techniques. One long-term goal of doing this could be to design computational language agents that are capable of teaching the language to new students using the method. In this paper, we aimed to investigate the feasibility of training an agent to deliver a Te Ataarangi lesson by carefully formulating the problem, defining the requirements of the agent and then testing out the components one by one. At the end, we would like to demonstrate that our computational model works \textit{because} it follows the principles of the original method.

Following this line of thinking, we observe that a typical Te Ataarangi lesson generally takes the form of a scalar-implicature game, wherein the teacher uses cuisenaire rods and hand gestures to specify a world state, or a \textit{context} ($c$) for discussion. They then give an \textit{utterance} ($u$) corresponding to that context, and over time the job of the student is to identify the relevant grammatical rules and vocabulary so that they can generalise the given utterance to other contexts. The student can then demonstrate their knowledge by carefully constructing their own contexts, and proving they have the correct utterance to the teacher.

The lesson bears a resemblance to a Socratic dialogue and incorporates aspects of pragmatics. Since the teacher must think about what the student knows or does not know about the language, and construct examples in order to falsify incorrect assumptions, or to add missing context that may also be leading the student astray. We therefore propose to model the lesson with a pair of Rational Speech Actors according to the Rational Speech Act \cite{frank2012predicting,goodman2016pragmatic} framework. We define two Rational Speech Actors, one each for the teacher and the student. Both agents implement a literal listener, a pragmatic speaker and a pragmatic listener. The literal listener is a model that defines a mapping from contexts ($c$) to utterances ($u$). The teacher agent is pre-trained, such that it has prior knowledge of the rules of the language. The student agent then learns through a dialogue with the teacher agent. The dialogue is set up in the following way:

\begin{itemize}
  \item The teacher selects an (context, utterance) pair and provides it to the student by way of demonstration.
  \item The student then selects an (context, utterance) pair and conjectures an utterance for that state for the teacher to evaluate.
  \item The student receives feedback from the teacher, either by way of correction, or the conversation may move on indicating that the selection was correct. The feedback is provided as an additional (context, utterance) pair.
  \item This loop continues until the teacher is satisfied that the student has acquired the required knowledge.
\end{itemize}

The contexts under consideration are defined as an arrangement of cuisenaire rods. Each rod has a colour and a length, and they are sampled at random and arranged into a row. From the sample, a selection is made and the utterance is constructed to accurately identify the selected rods among their neighbours. In English, utterances therefore look may look like ``The red rod'', ``The blue rod on the left'', or ``All of the rods'' if all of the rods in the context are selected.

Throughout this work, we operate under the assumption that we are working with an under-resourced language. So we place an emphasis on methods that could reasonably be applied by a language community in the absence of large training datasets containing millions of tokens. Te reo Māori is used as the language for all experiments, although it would be easy to run these experiments in any language, and would require the hand labelling of no more than 1,000 examples. As a speaker of the Māori language, I was able to prepare the dataset for these experiments alone, in only a few hours.

We begin the project by training a recurrent neural network to solve this scalar implicature problem, using a small training dataset. Initially, it was proposed to use a small dataset for parsimony reasons, as the task as designed resembles a very early beginner Māori language lesson, requiring minimal prior knowledge of the language. This problem proved to be more difficult than initially anticipated, and so various strategies, such as data augmentation were attempted in order to pre-train a competent teacher agent for this task.


\section{Related Work}
% Various works have studied linguistic cues related to deception, as well as similar phenomena like betrayal, in the context of games. 

Niculae et al. \cite{diplomacy} developed a framework for analyzing evolving dialogues in the game of \textit{Diplomacy}: a war-themed strategy game where players converse to form alliances, but must ultimately betray their allies to gain territory and win the game. In particular, they found that imminent betrayal was signaled by sudden, but slight, increases in positive sentiment and politeness on the part of the betrayer. The use of planning discourse markers was also found to correlate with betrayal, with betrayees typically using them more often than betrayers just before the deception.

% In recent work, Ibraheem et al. \cite{mafia} studied deception in the game of \textit{Mafia} where players are assigned a mafioso or a bystander role, with the objective being to either to learn the identities of all the mafiosos (for the bystanders) or to eliminate all the bystanders (for the mafiosos). Mafiosos are thus highly incentized to deceive the bystanders, whereas the bystanders are incentized to play an honest role. Ibraheem et al. applied two approaches to this problem: a standard BERT-based classifier that only used the utterances of the player to be classified, and a second auxiliary approach that also accounted for prior utterances from all players. They found that accounting for past utterances improved performance, and suggested that linguistic behaviour like referring to other players (especially for elimination), and asking for suggestions on how to eliminate were stronger indicators of mafiosos, whereas aspects of confusion may be more strongly correlated with bystanders. 

This work provides insights into the subtle cues associated with deception. However the setting of a game is highly constrained: the environment is highly simplified compared to the real world. We expect that finding signs of deception in a free-form and open world, such as in the Enron case, will be more difficult. 

% Tie this in somehow:
%The italian court cases are from 2013, the diplomacy paper from 2015, too long to explain connection and differences

Several works have analyzed the Enron dataset in the past, specifically applying different models of deception. In particular, Gupta et al. \cite{enron_deception_gupta} and Keila et al. \cite{enron_deception_keila} both characterized deception by a reduced usage of first-person pronouns and exception words, and an increased usage of negative emotions words and action verbs. They each applied SVD to the emails, and found these latent factors in deceptive emails.

In other work, Eckhaus et al. \cite{hubris} investigated the use of words specifically linked to hubris. They found that among the executives eventually convicted, their usage of hubristic terms increased in frequency prior to the collapse. In this way, they illustrated a trend possibly correlated to deception, which suggests the potential existence of predictive features that could be used in other circumstances to identify signs of deception or deterioration. 

% As fraudulent practices at Enron had been going on for years if not decades, there is no specific ``deception moment''. We focus on the brief period that precedes the bankruptcy, as language might be likely to change around this time of reckoning, and deception of shareholders and the public was at its conscious peak. %The first signs of doubt appeared in the public space in the beginning of 2001 \cite{fortune}, leading up to the bankruptcy filing at the end of the same year (on December 2nd). We take October 2001 to be the beginning of the collapse, since it is at that time that the signs were apparent to the public (Enron's stock price started decreasing sharply, never to recover(CITATIONS)) and simultaneously, internal deception was provably practiced (Enron's legal council had requested auditors to destroy documents \cite{andersen-docs}). We therefore turn our attention to the beginning of October 2001. 

Motivated by this work, we apply a similar approach but with the deception features identified by Niculae et al. As fraudulent practices at Enron had been going on for years, there was no specific ``deception moment''. However, we observe the trends of positive sentiment, politeness, and planning discourse before and during the collapse to see whether there is a difference between individuals who were and were not convicted of crimes. Our hypothesis is that these markers of deception are transferable to a different real-life context. 

To further explore the space of good predictive features for deception at Enron, we trained linear classifiers to extract tokens that are most useful to predicting a fraudster. 


\section{Method}
\subsection{Dataset}

We use the Enron Email Dataset \cite{enron2015email} provided by Dr. William W. Cohen, specifically, the May 7, 2015 version of the dataset. The dataset contains 517,401 emails from employees at Enron that were obtained by the Federal Energy Regulatory Commission during its investigation of Enron's collapse. 

As this dataset only contains emails and little metadata, information about the authors was supplemented using additional sources. A \textit{New York Times} \cite{nytimes} archive was used to identify individuals who were convicted of crimes related to the collapse. Of these individuals, only four had emails present in the dataset, namely Kenneth Lay, Jeffrey Skilling, David Delainey, and John Forney. We shall refer to these individuals as persons-of-interest (POIs) from this point forward.

\begin{table}[t]
    \centering
    \caption{The breakdown of emails in our dataset including the number of emails in the training, validation, and test sets, and the number of POI, executive, and normal employee emails in each.}
    \label{tab:dataset_breakdown}
    \resizebox{0.48\textwidth}{!}{%
    \begin{tabular}{|l||r|r|r|r|}
        \hline
         & \textbf{POI} & \textbf{Executive} & \textbf{Normal} & \textbf{Total} \\
        \hline
        \hline
        Training & 670 & 1,807 & 5,467 & \textbf{7,944} \\
        \hline
        Validation & 167 & 450 & 1,369 & \textbf{1,986} \\
        \hline
        Test & 272 & 774 & 2,264 & \textbf{3,310} \\
        \hline
        \textbf{Total} & \textbf{1,109} & \textbf{3,031} & \textbf{9,100} & \textbf{13,240} \\
        \hline
    \end{tabular}%
    }
\end{table}

As all the POIs were executives at Enron, further information was collected to identify other executives in the dataset who were not eventually convicted of fraud. This was to ensure a fair comparison, and to add confidence that any features associated with POIs were not more broadly applicable to other executives. A dataset \cite{enron_financial_dataset} of financial information, while not containing any explicit information on employees' positions, was used to identify individuals similar to executives. For the purposes of this project, anyone with a salary above $\$200,000$ USD per year was considered an executive.
%The dataset \cite{enron_financial_dataset} of financial information does not contain explicit information on employees' positions, but the salary information was used to identify individuals similar to executives. For the purposes of this project, anyone with a salary above $\$200,000$ USD per year was considered as an executive.



%A dataset of financial information was found \cite{enron_financial_dataset}, and while this dataset did not contain explicit information on employees' positions, the salary information was used to identify individuals similar to executives. 

A subset of the emails was used which consisted of emails from 22 authors: the 4 POIs (Lay, Skilling, Delainey, and Forney), 5 other executives who were not convicted of crimes, as well as 13 other ``normal" employees at Enron who were neither POIs nor executives. This uneven corporate rank distribution was chosen to reflect a more accurate employee-to-executive ratio to what might be expected in other companies. Overall, roughly 700 emails were taken from each person to ensure each author was similarly represented. This, however, was only applied to executives and normal employees, to ensure the fraction of POI emails would remain low. %This was to reflect the expectation that in a normal organization, the number of frauds would be expected to be low. 
In particular, our email subset contained 13,240 emails with 8.3\% of emails from POIs, 22.9\% of emails from execs, and 68.8\% of emails from normal employees. 

In terms of the authors chosen, the executive emails were taken from Allen, Kitchen, Lavorato, Shankman, and Shapiro. For the normal employee emails, they were taken from Bass, Dasovich, Davis, Germany, Jones, Lenhart, Mann, Nemec, Perlingiere, Rogers, Scott, Shackleton, and Symes. These authors were chosen since they had a large number of emails in the original Enron dataset.

% Further cross-referencing with an article of the \textit{Financial Times} \cite{enron_financial_dataset}, which contains the income of many Enron employees, allowed us to annotate many of the email authors according to their status in the company. We refer to the employees with incomes above \$200K as execs, sometimes also differentiating higher execs (income over \$300K). This is important to avoid social status bias in politeness classification (cite someone who says low politeness correlates with high status?????). Authors for whom we were unable to find income information were assumed to be of ``normal'' income. 

\subsection{Data Cleaning}

% One major data processing step was to annotate the emails with extra information about the authors. Cross-referencing with an article in the \textit{New York Times} \cite{nytimes} which summarizes the outcomes of the Enron trials allowed us to identify authors that were convicted of, or plead guilty to, fraud. Since the Enron data set is not a complete collection of all emails written by Enron employees over the years, there were only four authors (Lay, Delainey, Skilling, Forney) who were convicted of fraud. Throughout our work, they are referred to as POI (people of interest). Further cross-referencing with an article of the \textit{Financial Times} \cite{enron_financial_dataset}, which contains the income of many Enron employees, allowed us to annotate many of the email authors according to their status in the company. We refer to the employees with incomes above \$200K as execs, sometimes also differentiating higher execs (income over \$300K). This is important to avoid social status bias in politeness classification (cite someone who says low politeness correlates with high status?????). Authors for whom we were unable to find income information were assumed to be of ``normal'' income. 

Emails were gathered using the email addresses of the relevant individuals. Any email address that contained the name of an individual was used for that particular author.

Since our focus was on the language of the authors themselves, careful selection was done to ensure that only emails written by the actual authors were kept. Forwarded and replied text was removed, as well as messages written by the secretaries or assistants of the individuals. This was done programmatically using the names of secretaries discovered in random samples of the dataset. While it is still possible that some emails may have been written by other individuals, for the emails of the POIs all emails were manually inspected to ensure they were all written by the sender. 

Additional data cleaning involved removing empty and duplicate emails. Finally, only emails consisting of 5 words or more were kept, and all emails sent from before 1999 were also dropped since they only accounted for about 100 emails.

\subsection{Modeling}

Following Niculae et al., we model the positive sentiment, politeness, and planning discourse markers in the emails.

For sentiment, the \textit{SiEBERT} \cite{sentimentModel} model from Huggingface was used. It is a fine-tuned checkpoint of RoBERTa-large, originally evaluated on diverse datasets coming from reviews, tweets and other similar sources. 

For the politeness model, a pre-trained politeness model trained on the Wikipedia Politeness Corpus was used provided in the ConvoKit package.

To study planning discourse, we collated a list of individual (single-word) and phrasal (bi-gram) planning discourse markers. The individual markers we chose were "shall", "going", "planning", "intend", "aim", "hope" and "expect", and the phrasal markers we chose were "plan to", "aim to", "looking forward", "in preparation" and "prepare for". For each class, and each month, we totaled the number of planning discourse markers we were able to match and divided by the number of sentences in the sample, as identified by NLTK. This gave us the percentage of planning discourse markers per sentence per month across each of the 3 employee classes under consideration.

Finally, for training our own classifiers, we tested a mix of 24 different logistic regression and naïve Bayes models to predict whether an email was written by a POI. These models varied in the text preprocessing step or in the hyperparamter value (smoothing parameter for naïve Bayes, and regularization strength for logistic regression). 

While these models are only linear classifiers, they were chosen due to their interpretability since the feature weights and probability values can measure the relative importance of the different words in the predictions.

% Note that for the naïve Bayes models, the \texttt{ComplementNB} model was used in scikit-learn since this variant is better suited to imbalanced datasets, like the one here. 

\subsection{Text Preprocessing}

% All emails were cleaned, as described above, before further processing.

For the sentiment analysis, politeness, and planning discourse markers, the emails were passed as-is to the model pipelines. 

For classification, entities for organizations, locations, and people names were masked in the emails. This was done to avoid the models from learning Enron-specific features in the emails, as the goal was to find linguistic cues that were as generalizable as possible to other deception/fraud domains. Specifically, all names were replaced by ``Steve'', all organizations by ``Apple'' and all locations by ``Cupertino''.

Emails were tokenized and lemmatized using NLTK. Stopwords were removed using NLTK's list of stopwords, and standard punctuation and digits were removed. Experiments involving stemming and bigrams were completed, however these models did not perform as well as unigram models.

Words were vectorized using TF-IDF. Other vectorization schemes including embeddings and other pre-trained methods were not attempted in order to preserve the explainability of the features. 

\subsection{Evaluation}

The classifiers were trained on the training set, and hyperparameters and the choice of best model was based on the F1 score of the POI class on the validation set. The F1 score was used due to the heavily imbalanced nature of the dataset. For the training/validation/test split, see Table \ref{tab:dataset_breakdown}.

The final evaluation reported in Table \ref{tab:best_model_metrics} is based on the precision, recall, and F1 score of the POI class on the test set.


\section{Results}
After data augmentation, we had 778,001 training samples 124,763 samples in the dev set and 101,314 samples were used for testing. These examples were used to train a range of transformer, RNN and LSTM models.

We were unable to obtain a model that was competent at the target task with the problem formulation we described.


\section{Discussion and Conclusion}
% \textbf{What conclusions can be drawn from your experiments? Was your initial hypothesis verified? What are the limitations of your work, and how could it be extended?}

% Sentiment classifier (hugging face ref) to test the hypothesis that negative sentiment correlates with fraud... Not sure if this is relevant here... Will find place later.

Results in Figures \ref{fig:politeness} and \ref{fig:discourse} align well with the hypotheses in \cite{diplomacy}, namely that increased politeness and planning discourse markers are cues related to deception. This gives credence to the conjecture that these linguistic cues are indeed useful in more complex environments beyond games, and are general traits related to deceptive behaviour.

Most notably, it is interesting that planning discourse markers increase dramatically at the very beginning of the collapse and then subside, even as further events are unfolding, whereas politeness increases gradually up to the very collapse. This is an interesting addition to the results in \cite{diplomacy}. Diplomacy covers a short time span, whether measured by physical time or by number of events leading up to the deception. It is therefore not surprising that all deception markers overlap. In the case of Enron, the collapse was unfolding over a year, leaving more space between events, and therefore more time for deception markers to develop. %It would be interesting to look further into whether planning discourse markers regularly precede increased politeness in the course of deception. 

The fact that our results align with the ones in \cite{diplomacy}, though the alignment is not perfect, gives us some food for thought. %, as we labour in a very different setting altogether. 
In particular, since the emails analyzed are not directed at the targets of the deception (i.e., the shareholders and authorities), our findings suggest that an individual's general level of politeness increases when that individual engages in deception more generally. 

Furthermore, the results from our classifier training gives some interesting insights into the word usage of deceptive and non-deceptive actors. Most notable are the words with negative weights (therefore indicative of a non-POI email). There we observe words like ``fax" and ``email" which may be more associated with non-executive individuals. Additionally words like ``pm" suggest potential meetings which may be more commonplace for non-executive individuals.

These words may not be applicable to other settings, however additional work may be needed to confirm this.

\subsection{Limitations}

One major limitation of this work is that our classification of POIs only occurs based a single email, and does not account for prior context or email threads within its prediction. Doing so in such a situation could potentially give even more insightful cues in deception and potentially manipulation.
% Though our dataset contained over 13,000 emails, this may not be an appropriate size, and therefore represent appropriate difficulty, in detecting fraudulent actors in other settings. In this case, our approach was limited by the number of emails corresponding to the POIs, and more data could give better results.

% Aspects specific to the Enron dataset were also problematic. For example, almost all of the emails from Kenneth Lay's inbox were written by his assistant, which had to be discarded since they were not written by him. 
% This choice was made in order to preserve the linguistic qualities of the emails. However, it had several consequences, the obvious one being that we were left with very few emails from K. Lay, who was a major player in the Enron collapse, but also a top exec who rarely wrote his own emails. A message written by his assistant from his account was necessarily written at his behest. We have thus deprived ourselves of a large number of messages that would faithfully represent his intent. This is not a trivial loss, since deception is perpetrated both in content and in form (cite the italian papers or something ???) and we have restricted ourselves entirely to the form. 

Another limitation is that a large part of the deception at Enron was targeted at the general public. As such, interviews and public communications by POIs would have added valuable linguistic cues that are not included in our analysis. 

\subsection{Future work} 

% The discrepancy between the negative words identified as positive features for predicting POIs by our classifier, and our finding that negative sentiment is an area for more exploration. 
Incorporating models that can handle email context and email threads within their predictions would be an important step towards better understanding deception. 

Another interesting direction would be to test whether toxicity relates to fraud. This would be consistent with the results of our classifier and to some extent with \cite{hubris} seeing as toxicity and hubris can be linked.

% As mentioned above, another avenue for future work would be to investigate how the various deception markers are deployed relative to each other. Based on our meager evidence, we conjecture that an increase in planning discourse may be of shorter duration and may occur at the start of the deceptive practices, whereas politeness may increase throughout the deception. This also follows some common sense intuition. Given that planning discourse is more resource intensive and more deliberate, it makes sense that it be used parsimoniously and reluctantly. Politeness, on the other hand, can be thought of as a (conscious or subconscious) compensation technique, and therefore be used increasingly as the deception proceeds. 

\subsection{Conclusion}
We explored the possibility of identifying emails written by a fraudster based solely on their language using the Enron email dataset. We found that politness and planning discourse markers increased in frequency around the collapse, but sentiment did not appear related to deception. Our linear classifer acheieved good F1 score indicating that it is possible to identify POIs even among other executives. More research is required to validate the features identified by our models.

% In this paper we explore the possibility of identifying
% a fraudster based on their language. To this end, we consider the
% data set of emails of Enron employees, ranging from convicted
% offenders to regular employees. We look at fraud through the prism
% of deception, verifying if known markers of deception correlate
% with fraud. We also look for new potential language markers that
% could correlate to fraud.



% \section{Limitations}
% \input{sections/6-limitations}

% Entries for the entire Anthology, followed by custom entries
\bibliography{references}
\bibliographystyle{acl_natbib}

% \appendix

% \section{Example Appendix}
% \label{sec:appendix}

% This is a section in the appendix.

\end{document}
