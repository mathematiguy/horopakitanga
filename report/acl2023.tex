% This must be in the first 5 lines to tell arXiv to use pdfLaTeX, which is strongly recommended.
\pdfoutput=1
% In particular, the hyperref package requires pdfLaTeX in order to break URLs across lines.

\documentclass[11pt]{article}

% Remove the "review" option to generate the final version.
\usepackage{ACL2023}

% Standard package includes
\usepackage{times}
\usepackage{latexsym}

% For proper rendering and hyphenation of words containing Latin characters (including in bib files)
\usepackage[T1]{fontenc}
% For Vietnamese characters
% \usepackage[T5]{fontenc}
% See https://www.latex-project.org/help/documentation/encguide.pdf for other character sets

% This assumes your files are encoded as UTF8
\usepackage[utf8]{inputenc}

% This is not strictly necessary, and may be commented out.
% However, it will improve the layout of the manuscript,
% and will typically save some space.
\usepackage{microtype}

% This is also not strictly necessary, and may be commented out.
% However, it will improve the aesthetics of text in
% the typewriter font.
\usepackage{inconsolata}

\usepackage{amsmath,amssymb,amsfonts}
%\usepackage{ebgaramond-maths}
\usepackage{algorithmic}
\usepackage{graphicx, subcaption}
\usepackage{textcomp}
\usepackage{xcolor}
\usepackage{float}

% If the title and author information does not fit in the area allocated, uncomment the following
%
%\setlength\titlebox{<dim>}
%
% and set <dim> to something 5cm or larger.

\title{Detecting Deception in Enron Emails}

% Author information can be set in various styles:
% For several authors from the same institution:
% \author{Author 1 \and ... \and Author n \\
%         Address line \\ ... \\ Address line}
% if the names do not fit well on one line use
%         Author 1 \\ {\bf Author 2} \\ ... \\ {\bf Author n} \\
% For authors from different institutions:
% \author{Author 1 \\ Address line \\  ... \\ Address line
%         \And  ... \And
%         Author n \\ Address line \\ ... \\ Address line}
% To start a seperate ``row'' of authors use \AND, as in
% \author{Author 1 \\ Address line \\  ... \\ Address line
%         \AND
%         Author 2 \\ Address line \\ ... \\ Address line \And
%         Author 3 \\ Address line \\ ... \\ Address line}

\author{David Hobson \and Caleb Moses \and Svetla Vassileva \\
  \texttt{\{david.hobson, caleb.moses, svetla.vassileva\}@mail.mcgill.ca} \\
  \textit{dept. of Computer Science, McGill University, Montreal, Canada}}


\begin{document}
\pagestyle{plain}
\thispagestyle{plain}

\maketitle
\begin{abstract}
In this paper we explore the possibility of identifying a fraudulent actor based on their language. We consider the email dataset of Enron employees which includes emails from convicted offenders and regular employees. We look at fraud through the prism of deception, verifying if known markers of deception correlate with fraud. We also look for new potential language markers that could correlate to fraud. The code is available on GitHub\footnote{\url{https://github.com/mathematiguy/enron-nlp-analysis}}.
\end{abstract}

%To this end,
% ranging from convicted offenders to regular employees

\section{Introduction}
Identifying deception is a difficult task \cite{mafia}. While humans usually act collaboratively, people can be incentivized to lie or deceive others for many reasons. %, whether to avoid confrontation or conflict, to avoid losing face, or for nefarious reasons including personal gain or the achievement of some criminal objective. 
Previous studies have shown that people can struggle with identifying lies from truths \cite{mafia}. Even trained law enforcement professionals are only slightly better than chance at detecting deception \cite{enron_deception_gupta}.

Many works have studied and identified different computational linguistic features which are correlated with deceptive actions in controlled environments, for example in games like \textit{Diplomacy} \cite{diplomacy} and \textit{Mafia} \cite{mafia}. In this work we speculate as to whether such deceptive patterns apply to more complex real-world settings, such as fraud detection.

Among the most famous fraud cases is the collapse of Enron, the American energy company that filed for bankruptcy in 2001 after widespread fraud within the company was revealed by the media. In the aftermath of the scandal, several Enron executives and employees were convicted or pleaded guilty of fraud and related crimes.

Since the collapse, some of the emails from Enron employees discovered in the course of the court case have been made public. This dataset includes emails leading up to the events and in particular from high-profile executives who were ultimately convicted \cite{enron_dataset}. This Enron email dataset is a unique testbed for the study of markers of fraud and deception in a real-world scenario.

Combining this with the prior research studying deception in games, the goal of this project was to analyze the Enron email dataset, and observe whether the same markers of deception that are displayed in game environments are also present within the emails of Enron employees who were eventually convicted of fraud.

We further train our own classifiers to identify emails written by fraudsters, and assess the linguistic features it finds most useful. 

%\textbf{What is the hypothesis that you test and how do you go about doing so?}
%
%To this end, we... 
%
%It turns out that...
%
% This project provided an exciting opportunity to dive into this ``true crime'' world and investigate the nature of fraud from a language analysis prospective. Fraud presupposes deception and we therefore leaned into identifying deception from text. 

% We then proceed to search for other potential markers. The only one that stood out was negative sentiment. We explore the correlation between negative sentiment and fraud. 


% was not statistically significant. 


% It turns out that, indeed, none of the deception markers identified in [danescu] can be applied transversally to the deception in Enron. 




\section{Related Work}
\input{sections/related_work}

\section{Method}
\input{sections/method}

\section{Results}
\input{sections/results}

\section{Discussion and Conclusion}
% \textbf{What conclusions can be drawn from your experiments? Was your initial hypothesis verified? What are the limitations of your work, and how could it be extended?}

% Sentiment classifier (hugging face ref) to test the hypothesis that negative sentiment correlates with fraud... Not sure if this is relevant here... Will find place later.

Results in Figures \ref{fig:politeness} and \ref{fig:discourse} align well with the hypotheses in \cite{diplomacy}, namely that increased politeness and planning discourse markers are cues related to deception. This gives credence to the conjecture that these linguistic cues are indeed useful in more complex environments beyond games, and are general traits related to deceptive behaviour.

Most notably, it is interesting that planning discourse markers increase dramatically at the very beginning of the collapse and then subside, even as further events are unfolding, whereas politeness increases gradually up to the very collapse. This is an interesting addition to the results in \cite{diplomacy}. Diplomacy covers a short time span, whether measured by physical time or by number of events leading up to the deception. It is therefore not surprising that all deception markers overlap. In the case of Enron, the collapse was unfolding over a year, leaving more space between events, and therefore more time for deception markers to develop. %It would be interesting to look further into whether planning discourse markers regularly precede increased politeness in the course of deception. 

The fact that our results align with the ones in \cite{diplomacy}, though the alignment is not perfect, gives us some food for thought. %, as we labour in a very different setting altogether. 
In particular, since the emails analyzed are not directed at the targets of the deception (i.e., the shareholders and authorities), our findings suggest that an individual's general level of politeness increases when that individual engages in deception more generally. 

Furthermore, the results from our classifier training gives some interesting insights into the word usage of deceptive and non-deceptive actors. Most notable are the words with negative weights (therefore indicative of a non-POI email). There we observe words like ``fax" and ``email" which may be more associated with non-executive individuals. Additionally words like ``pm" suggest potential meetings which may be more commonplace for non-executive individuals.

These words may not be applicable to other settings, however additional work may be needed to confirm this.

\subsection{Limitations}

One major limitation of this work is that our classification of POIs only occurs based a single email, and does not account for prior context or email threads within its prediction. Doing so in such a situation could potentially give even more insightful cues in deception and potentially manipulation.
% Though our dataset contained over 13,000 emails, this may not be an appropriate size, and therefore represent appropriate difficulty, in detecting fraudulent actors in other settings. In this case, our approach was limited by the number of emails corresponding to the POIs, and more data could give better results.

% Aspects specific to the Enron dataset were also problematic. For example, almost all of the emails from Kenneth Lay's inbox were written by his assistant, which had to be discarded since they were not written by him. 
% This choice was made in order to preserve the linguistic qualities of the emails. However, it had several consequences, the obvious one being that we were left with very few emails from K. Lay, who was a major player in the Enron collapse, but also a top exec who rarely wrote his own emails. A message written by his assistant from his account was necessarily written at his behest. We have thus deprived ourselves of a large number of messages that would faithfully represent his intent. This is not a trivial loss, since deception is perpetrated both in content and in form (cite the italian papers or something ???) and we have restricted ourselves entirely to the form. 

Another limitation is that a large part of the deception at Enron was targeted at the general public. As such, interviews and public communications by POIs would have added valuable linguistic cues that are not included in our analysis. 

\subsection{Future work} 

% The discrepancy between the negative words identified as positive features for predicting POIs by our classifier, and our finding that negative sentiment is an area for more exploration. 
Incorporating models that can handle email context and email threads within their predictions would be an important step towards better understanding deception. 

Another interesting direction would be to test whether toxicity relates to fraud. This would be consistent with the results of our classifier and to some extent with \cite{hubris} seeing as toxicity and hubris can be linked.

% As mentioned above, another avenue for future work would be to investigate how the various deception markers are deployed relative to each other. Based on our meager evidence, we conjecture that an increase in planning discourse may be of shorter duration and may occur at the start of the deceptive practices, whereas politeness may increase throughout the deception. This also follows some common sense intuition. Given that planning discourse is more resource intensive and more deliberate, it makes sense that it be used parsimoniously and reluctantly. Politeness, on the other hand, can be thought of as a (conscious or subconscious) compensation technique, and therefore be used increasingly as the deception proceeds. 

\subsection{Conclusion}
We explored the possibility of identifying emails written by a fraudster based solely on their language using the Enron email dataset. We found that politness and planning discourse markers increased in frequency around the collapse, but sentiment did not appear related to deception. Our linear classifer acheieved good F1 score indicating that it is possible to identify POIs even among other executives. More research is required to validate the features identified by our models.

% In this paper we explore the possibility of identifying
% a fraudster based on their language. To this end, we consider the
% data set of emails of Enron employees, ranging from convicted
% offenders to regular employees. We look at fraud through the prism
% of deception, verifying if known markers of deception correlate
% with fraud. We also look for new potential language markers that
% could correlate to fraud.



\section{Statement of Contributions}

All team members contributed equally to the design of the project. D. Hobson and C. Moses processed the emails. S. Vassileva performed the sentiment analysis, D. Hobson tackled politeness and trained the classifiers and C. Moses analysed the discourse markers. All team members contributed to writing the report. 


%%%%%%%
%Everyone contributed to design of project.
%
%David and Caleb did email processing.
%
%Svetla did sentiment.
%David did politeness.
%Caleb did discourse markers.
%
%David trained classifiers.
%
%Everyone contributed to writing.

% Entries for the entire Anthology, followed by custom entries
\bibliography{references}
\bibliographystyle{acl_natbib}

% \appendix

% \section{Example Appendix}
% \label{sec:appendix}

% This is a section in the appendix.

\end{document}
