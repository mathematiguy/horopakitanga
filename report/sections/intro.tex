Identifying deception is a difficult task \cite{mafia}. While humans usually act collaboratively, people can be incentivized to lie or deceive others for many reasons. %, whether to avoid confrontation or conflict, to avoid losing face, or for nefarious reasons including personal gain or the achievement of some criminal objective. 
Previous studies have shown that people can struggle with identifying lies from truths \cite{mafia}. Even trained law enforcement professionals are only slightly better than chance at detecting deception \cite{enron_deception_gupta}.

Many works have studied and identified different computational linguistic features which are correlated with deceptive actions in controlled environments, for example in games like \textit{Diplomacy} \cite{diplomacy} and \textit{Mafia} \cite{mafia}. In this work we speculate as to whether such deceptive patterns apply to more complex real-world settings, such as fraud detection.

Among the most famous fraud cases is the collapse of Enron, the American energy company that filed for bankruptcy in 2001 after widespread fraud within the company was revealed by the media. In the aftermath of the scandal, several Enron executives and employees were convicted or pleaded guilty of fraud and related crimes.

Since the collapse, some of the emails from Enron employees discovered in the course of the court case have been made public. This dataset includes emails leading up to the events and in particular from high-profile executives who were ultimately convicted \cite{enron_dataset}. This Enron email dataset is a unique testbed for the study of markers of fraud and deception in a real-world scenario.

Combining this with the prior research studying deception in games, the goal of this project was to analyze the Enron email dataset, and observe whether the same markers of deception that are displayed in game environments are also present within the emails of Enron employees who were eventually convicted of fraud.

We further train our own classifiers to identify emails written by fraudsters, and assess the linguistic features it finds most useful. 

%\textbf{What is the hypothesis that you test and how do you go about doing so?}
%
%To this end, we... 
%
%It turns out that...
%
% This project provided an exciting opportunity to dive into this ``true crime'' world and investigate the nature of fraud from a language analysis prospective. Fraud presupposes deception and we therefore leaned into identifying deception from text. 

% We then proceed to search for other potential markers. The only one that stood out was negative sentiment. We explore the correlation between negative sentiment and fraud. 


% was not statistically significant. 


% It turns out that, indeed, none of the deception markers identified in [danescu] can be applied transversally to the deception in Enron. 


