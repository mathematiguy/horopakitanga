% Various works have studied linguistic cues related to deception, as well as similar phenomena like betrayal, in the context of games. 

Niculae et al. \cite{diplomacy} developed a framework for analyzing evolving dialogues in the game of \textit{Diplomacy}: a war-themed strategy game where players converse to form alliances, but must ultimately betray their allies to gain territory and win the game. In particular, they found that imminent betrayal was signaled by sudden, but slight, increases in positive sentiment and politeness on the part of the betrayer. The use of planning discourse markers was also found to correlate with betrayal, with betrayees typically using them more often than betrayers just before the deception.

% In recent work, Ibraheem et al. \cite{mafia} studied deception in the game of \textit{Mafia} where players are assigned a mafioso or a bystander role, with the objective being to either to learn the identities of all the mafiosos (for the bystanders) or to eliminate all the bystanders (for the mafiosos). Mafiosos are thus highly incentized to deceive the bystanders, whereas the bystanders are incentized to play an honest role. Ibraheem et al. applied two approaches to this problem: a standard BERT-based classifier that only used the utterances of the player to be classified, and a second auxiliary approach that also accounted for prior utterances from all players. They found that accounting for past utterances improved performance, and suggested that linguistic behaviour like referring to other players (especially for elimination), and asking for suggestions on how to eliminate were stronger indicators of mafiosos, whereas aspects of confusion may be more strongly correlated with bystanders. 

This work provides insights into the subtle cues associated with deception. However the setting of a game is highly constrained: the environment is highly simplified compared to the real world. We expect that finding signs of deception in a free-form and open world, such as in the Enron case, will be more difficult. 

% Tie this in somehow:
%The italian court cases are from 2013, the diplomacy paper from 2015, too long to explain connection and differences

Several works have analyzed the Enron dataset in the past, specifically applying different models of deception. In particular, Gupta et al. \cite{enron_deception_gupta} and Keila et al. \cite{enron_deception_keila} both characterized deception by a reduced usage of first-person pronouns and exception words, and an increased usage of negative emotions words and action verbs. They each applied SVD to the emails, and found these latent factors in deceptive emails.

In other work, Eckhaus et al. \cite{hubris} investigated the use of words specifically linked to hubris. They found that among the executives eventually convicted, their usage of hubristic terms increased in frequency prior to the collapse. In this way, they illustrated a trend possibly correlated to deception, which suggests the potential existence of predictive features that could be used in other circumstances to identify signs of deception or deterioration. 

% As fraudulent practices at Enron had been going on for years if not decades, there is no specific ``deception moment''. We focus on the brief period that precedes the bankruptcy, as language might be likely to change around this time of reckoning, and deception of shareholders and the public was at its conscious peak. %The first signs of doubt appeared in the public space in the beginning of 2001 \cite{fortune}, leading up to the bankruptcy filing at the end of the same year (on December 2nd). We take October 2001 to be the beginning of the collapse, since it is at that time that the signs were apparent to the public (Enron's stock price started decreasing sharply, never to recover(CITATIONS)) and simultaneously, internal deception was provably practiced (Enron's legal council had requested auditors to destroy documents \cite{andersen-docs}). We therefore turn our attention to the beginning of October 2001. 

Motivated by this work, we apply a similar approach but with the deception features identified by Niculae et al. As fraudulent practices at Enron had been going on for years, there was no specific ``deception moment''. However, we observe the trends of positive sentiment, politeness, and planning discourse before and during the collapse to see whether there is a difference between individuals who were and were not convicted of crimes. Our hypothesis is that these markers of deception are transferable to a different real-life context. 

To further explore the space of good predictive features for deception at Enron, we trained linear classifiers to extract tokens that are most useful to predicting a fraudster. 
